%%%%%%%%%%%%%%%%%%%%%%%%%%%%%%%%%%%%
% Header                           %
%%%%%%%%%%%%%%%%%%%%%%%%%%%%%%%%%%%%
% 
% This is the documentation for the
% stmmath package
%
% Usage
%  - Compile using 'arara stmmath.tex'
% 
% Revisions: 2019-10-27 Martin Raedel <martin.raedel@dlr.de>
%                       Initial draft
%               
% Contact:   Martin Raedel,  martin.raedel@dlr.de
%            DLR Lightweight Systems
%          
%                                 __/|__
%                                /_/_/_/  
%            www.dlr.de/fa/en      |/ DLR
%
% Copyright (C) 2019-... DLR Lightweight Systems
% 
%%%%%%%%%%%%%%%%%%%%%%%%%%%%%%%%%%%%
% Content                          %
%%%%%%%%%%%%%%%%%%%%%%%%%%%%%%%%%%%%

% --------------------------- 
% Documentclass
% ---------------------------

\documentclass[%
  type=article,%
  layout=koma,%
  date=true,%
  hyperref=true,%
  listings=true,%
  math=true,%
]{stmtext}

% ---------------------------
% Build automation
% ---------------------------

% arara: pdflatex: {shell: yes, synctex: yes, interaction: nonstopmode}
% arara: pdflatex: {shell: yes, synctex: yes, interaction: nonstopmode}
% arara: clean: { extensions: [ acn, acr, alg, aux, auxlock, bbl, bcf, blg, dpth, dvi, glg, glo, gls, idx, ilg, ind, ist, kate-swp, lock, lof, log, lol, lot, mw, nlo, out, ps, run.xml, slg, slg*, syg, syg*, syi, syi*, synctex, synctex.gz, tex.backup, tex.kate-swp, toc*, user.adi ] }

% ---------------------------
% Packages
% ---------------------------

\usepackage[T1]{fontenc}
\usepackage[utf8]{inputenc}
\usepackage{tabularx}

% ---------------------------
% Doc info
% ---------------------------

\author{Martin R\"{a}del}
\title{stmmath package description}
\subtitle{Copyright \copyright{} \the\year{} DLR SY STM\\v\formatdate[versiondatestyle]{\DTMToday}}
\date{\today}

%%%%%%%%%%%%%%%%%%%%%%%%%%%%%%%%%%%%
% Document                         %
%%%%%%%%%%%%%%%%%%%%%%%%%%%%%%%%%%%%

\begin{document}

\maketitle

\begin{abstract}
These are the math definitions for \texttt{stmlatex}. It is build upon the \href{https://ctan.org/pkg/amsmath}{amsmath} package.
\end{abstract}

\tableofcontents

\section{Commands}

\subsection{Operators}

\begin{tabularx}{\linewidth}{lXc}
\verb+\dev+ & Deviatoric & $\dev$\\
\verb+\dif+ & Infinitesimal differential & $\dif$\\
\verb+\divergenceoperator+ & Quantity of a vector field & $\divergenceoperator$\\
\verb+\erf+ & Error function & $\erf$\\
\verb+\sign+ & Signum function & $\sign$\\
\verb+\sph+ & Spherical & $\sph$\\
\verb+\spur+ & Trace & $\spur$\\
\verb+\Grad+ & Gradient w.r.t. material coordinates & $\Grad$\\
\verb+\grad+ & Gradient w.r.t. spatial coordinates & $\grad$
\end{tabularx}

\subsection{Symbols}

\begin{tabularx}{\linewidth}{Xc}
\verb+\minus+ & $\minus$\\
\verb+\curveplus+ & $\curveplus$\\
\verb+\rightplus+ & $\rightplus$\\
\verb+\upplus+ & $\upplus$
\end{tabularx}

\section{Commands}

There are additional commands available which require parameters. They are defined dependent of the symbols used in \texttt{stmglossaries}.

\section{Environments}

\newpage

\appendix

\newpage
\section{The code}

\subsection{\protect\texttt{stmmath.sty}}

\lstinputlisting[
  style=texpackagedocstyle,%
]{../../tex/latex/stmmath/stmmath.sty}

\subsection{\protect\texttt{stmmathbase.sty}}

\lstinputlisting[
  style=texpackagedocstyle,%
]{../../tex/latex/stmmath/stmmathbase.sty}

\subsection{\protect\texttt{stmmathitems.sty}}

\lstinputlisting[
  style=texpackagedocstyle,%
]{../../tex/latex/stmmath/stmmathitems.sty}

\end{document}
