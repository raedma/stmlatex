%%%%%%%%%%%%%%%%%%%%%%%%%%%%%%%%%%%%
% Header                           %
%%%%%%%%%%%%%%%%%%%%%%%%%%%%%%%%%%%%
% 
% This is the documentation for the
% stmconditionallox package
%
% Usage
%  - Compile using 'arara stmconditionallox.tex'
% 
% Revisions: 2019-10-27 Martin Raedel <martin.raedel@dlr.de>
%                       Initial draft
%               
% Contact:   Martin Raedel,  martin.raedel@dlr.de
%            DLR Composite Structures and Adaptive Systems
%          
%                                 __/|__
%                                /_/_/_/  
%            www.dlr.de/fa/en      |/ DLR
%
% Copyright (C) 2019-... DLR Composite Structures and Adaptive Systems
% 
%%%%%%%%%%%%%%%%%%%%%%%%%%%%%%%%%%%%
% Content                          %
%%%%%%%%%%%%%%%%%%%%%%%%%%%%%%%%%%%%

% --------------------------- 
% Documentclass
% ---------------------------

\documentclass[listof=totoc]{scrartcl}

% ---------------------------
% Build automation
% ---------------------------

% arara: pdflatex: {shell: yes, synctex: yes, interaction: nonstopmode}
% arara: pdflatex: {shell: yes, synctex: yes, interaction: nonstopmode}
% arara: pdflatex: {shell: yes, synctex: yes, interaction: nonstopmode}
% arara: remove: { patterns: [ '*.acn', '*.acr', '*.alg', '*.aux', '*.auxlock', '*.bbl', '*.bcf', '*.blg', '*.dpth', '*.dvi', '*.glg', '*.glo', '*.gls', '*.idx', '*.ilg', '*.ind', '*.ist', '*kate-swp', '*.lock', '*.lof', '*.log', '*.lol', '*.lot', '*.mw', '*.nlo', '*.out', '*.ps', '*.run.xml', '*.slg*', '*.syg*', '*.syi*', '*.synctex', '*.synctex.gz', '*.tex.backup', '*tex.kate-swp', '*.toc*', '*.user.adi*' ], recursive: yes }

% ---------------------------
% Packages
% ---------------------------

\usepackage[T1]{fontenc}
\usepackage[utf8]{inputenc}
\usepackage{graphicx}
\usepackage{stmdate}
\usepackage{stmlistings}
\usepackage[nonewpage]{stmconditionallox}

\usepackage{hyperref}

% ---------------------------
% Doc info
% ---------------------------

\author{Martin R\"{a}del}
\title{stmconditionallox package description}
\subtitle{Copyright \copyright{} \the\year{} DLR FA STM\\v\DTMsetdatestyle{versiondate}\DTMtoday}
\date{\today}

%%%%%%%%%%%%%%%%%%%%%%%%%%%%%%%%%%%%
% Document                         %
%%%%%%%%%%%%%%%%%%%%%%%%%%%%%%%%%%%%

\begin{document}

\maketitle

\begin{abstract}
These are definitions for conditional lists of float objects for \texttt{stmlatex}. It creates commands that only create lists in case the float object type is really used in the document. It is build upon the \href{https://ctan.org/pkg/totalcount}{totalcount} package.
\end{abstract}

\tableofcontents

\conditionallistoffigures  % Insert List of Figures if figures are present
\conditionallistoftables   % Insert List of Tables if tables are present
\conditionallistoflistings % Insert List of Listings if listings are present

\section{Usage}

\subsection{Commands}
% \subsection{Standard \protect\LaTeX}

There are standard \LaTeX commands to create lists of float objects:

\begin{itemize}
  \item \verb+\listoffigures+
  \item \verb+\listoftables+
\end{itemize}

Additionally it is possible to create a list of code listings with a macro from the \href{https://ctan.org/pkg/listings}{listings} package.

\begin{itemize}
  \item \verb+\lstlistoflistings+
\end{itemize}

% \subsection{\protect\texttt{stmconditionallox}}

With

\begin{verbatim}
\usepackage{stmconditionallox}
\end{verbatim}

a set of new commands is created:

\begin{itemize}
  \item \verb+\conditionallistoffigures+
  \item \verb+\conditionallistoftables+
  \item \verb+\conditionallistoflistings+
\end{itemize}

If these are used, the lists are only created, in case the specified float environments are really used in your document. Thus, it allows the creation of a standard preamble.

To properly create the lists, three compilation runs are required.

\subsection{Options}

\paragraph{Option \protect\textit{loadlistings} and \protect\textit{noloadlistingse}}

Set whether to load the \href{https://ctan.org/pkg/listings}{listings} package.

\begin{verbatim}
\usepackage[noloadlistings]{stmglossaries}
\end{verbatim}

\texttt{loadlistings} is the default and loads the package. It is used in case \texttt{noloadlistings} is not set explicitly.

In case \texttt{noloadlistings} is set, the command \verb+\conditionallistoflistings+ is not defined.

\paragraph{Option \protect\textit{newpage} and \protect\textit{nonewpage}}

Set whether to create a new page after each list.

\begin{verbatim}
\usepackage[nonewpage]{stmglossaries}
\end{verbatim}

\texttt{newpage} is the default and creates a new page. It is used in case \texttt{nonewpage} is not set explicitly.

\section{Example}

This section contains examples.

\subsection{Figures}

This is an example figure.

\begin{figure}[htbp]
\centering
\includegraphics[width=0.5\textwidth,height=0.3\textheight,keepaspectratio]{example-image-a}
\label{fig:example:figure:1}
\caption{Example figure}
\end{figure}

\subsection{Table}

This is an example table.

\begin{table}[htbp]
\centering
\begin{tabular}{ll}
Column 1 & Column 2
\end{tabular}
\label{tab:example:table:1}
\caption{Example table}
\end{table}

% \subsection{Listing}
% 
% \begin{lstlisting}[caption={Example listing}]
%     Some source code
% \end{lstlisting}

\newpage
\appendix

\section{The code}

\subsection{stmconditionallox.sty}

\lstinputlisting[breaklines,style=texcodestyle,nolol]{../../tex/latex/stmconditionallox/stmconditionallox.sty}

\end{document}
