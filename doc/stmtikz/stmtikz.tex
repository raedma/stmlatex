%%%%%%%%%%%%%%%%%%%%%%%%%%%%%%%%%%%%
% Header                           %
%%%%%%%%%%%%%%%%%%%%%%%%%%%%%%%%%%%%
% 
% This is the documentation for the
% stmtikz package
%
% Usage
%  - Compile using 'arara stmtikz.tex'
% 
% Revisions: 2019-10-27 Martin Raedel <martin.raedel@dlr.de>
%                       Initial draft
%               
% Contact:   Martin Raedel,  martin.raedel@dlr.de
%            DLR Composite Structures and Adaptive Systems
%          
%                                 __/|__
%                                /_/_/_/  
%            www.dlr.de/fa/en      |/ DLR
%
% Copyright (C) 2019-... DLR Composite Structures and Adaptive Systems
% 
%%%%%%%%%%%%%%%%%%%%%%%%%%%%%%%%%%%%
% Content                          %
%%%%%%%%%%%%%%%%%%%%%%%%%%%%%%%%%%%%

% --------------------------- 
% Documentclass
% ---------------------------

\documentclass{scrartcl}

% ---------------------------
% Build automation
% ---------------------------

% arara: pdflatex: {shell: yes, synctex: yes, interaction: nonstopmode}
% arara: pdflatex: {shell: yes, synctex: yes, interaction: nonstopmode}
% arara: remove: { patterns: [ '*.acn', '*.acr', '*.alg', '*.aux', '*.auxlock', '*.bbl', '*.bcf', '*.blg', '*.dpth', '*.dvi', '*.glg', '*.glo', '*.gls', '*.idx', '*.ilg', '*.ind', '*.ist', '*kate-swp', '*.lock', '*.lof', '*.log', '*.lol', '*.lot', '*.mw', '*.nlo', '*.out', '*.ps', '*.run.xml', '*.slg*', '*.syg*', '*.syi*', '*.synctex', '*.synctex.gz', '*.tex.backup', '*tex.kate-swp', '*.toc*', '*.user.adi*' ], recursive: yes }

% ---------------------------
% Packages
% ---------------------------

\usepackage[T1]{fontenc}
\usepackage[utf8]{inputenc}
\usepackage{enumitem}
\usepackage{stmdate}
\usepackage{stmlistings}
\usepackage[
  libraries=true,
  styles=true,
  externalization=true,
  globalexternalization=false,
  externalizationoutputfolder=ZZY_TikZ,
]{stmtikz}

\usepackage{hyperref}

% ---------------------------
% Doc info
% ---------------------------

\author{Martin R\"{a}del}
\title{stmtikz package description}
\subtitle{Copyright \copyright{} \the\year{} DLR FA STM\\v\DTMsetdatestyle{versiondate}\DTMtoday}
\date{\today}

%%%%%%%%%%%%%%%%%%%%%%%%%%%%%%%%%%%%
% Document                         %
%%%%%%%%%%%%%%%%%%%%%%%%%%%%%%%%%%%%

\begin{document}

\maketitle

\begin{abstract}
These are the tikz definitions for \texttt{stmlatex}. It is build upon the \href{https://ctan.org/pkg/tikz}{tikz} package.
\end{abstract}

\tableofcontents

\section{Usage - in the preamble}

\subsection{Load the whole \protect\texttt{stmtikz} package}

\subsubsection{Description}

This is an interface package which loads \texttt{tikz} and definitions commonly required throughout document creation.

By default the package loads

\begin{itemize}[noitemsep]
  \item \verb+stmtikzlibraries.sty+
  \item \verb+stmtikzstyles.sty+
  \item \verb+stmtikzexternalization.sty+
\end{itemize}

See \autoref{sec:usage:preamble:wholepackage:options} for options to change the default package behavior.

\subsubsection{Options}
\label{sec:usage:preamble:wholepackage:options}

\paragraph{Option \protect\textit{libraries}} 
\label{sec:usage:preamble:wholepackage:options:libraries}

This is a boolean option. Expected values are either \texttt{true} or \texttt{false}. It controls whether to load the standard libraries commonly required.

\begin{verbatim}
\usepackage[libraries=true|false]{stmtikz}
\end{verbatim}

\texttt{libraries=true} is the default. It is used in case \texttt{libraries=false} is not set explicitly.

\paragraph{Option \protect\textit{styles}} 
\label{sec:usage:preamble:wholepackage:options:styles}

This is a boolean option. Expected values are either \texttt{true} or \texttt{false}. It controls whether to load the predefined \texttt{tikz} styles.

\begin{verbatim}
\usepackage[styles=true|false]{stmtikz}
\end{verbatim}

\texttt{styles=true} is the default. It is used in case \texttt{styles=false} is not set explicitly.

\paragraph{Option \protect\textit{externalization}} 
\label{sec:usage:preamble:wholepackage:options:externalization}

This is a boolean option. Expected values are either \texttt{true} or \texttt{false}. It enables and disables the possibilities for the externalization of \texttt{tikzpicture}s.

\begin{verbatim}
\usepackage[externalization=true|false]{stmtikz}
\end{verbatim}

\texttt{externalization=true} is the default. It is used in case \texttt{externalization=false} is not set explicitly.

By default, global externalization is not set by the package. Thus it is required to use \texttt{\textbackslash tikzexternalenable} before the first picture, which is to be created with this feature. See \nameref{sec:usage:preamble:wholepackage:options:globalexternalization} for a possibility to allow a global externalization of throughout the document.

\paragraph{Option \protect\textit{externalizationoutputfolder}}
\label{sec:usage:preamble:wholepackage:options:outputfolder}

This option expects a string input. Do not add a slash at the end of the string.

With this option it is possible to define a output folder for all externalized \texttt{tikzpicture}s in case \nameref{sec:usage:preamble:wholepackage:options:externalization} has the value \texttt{true}. The folder location is set relative to the directory of the main \texttt{tex}-file.

\begin{verbatim}
\usepackage[externalizationoutputfolder=$FOLDERNAME$]{stmtikz}
\end{verbatim}

The default is \texttt{externalizationoutputfolder=ZZZ\_TikZ}.

\paragraph{Option \protect\textit{globalexternalization}}
\label{sec:usage:preamble:wholepackage:options:globalexternalization}

This is a boolean option.  Expected values are either \texttt{true} or \texttt{false}. 

By default externalization is not enabled for \texttt{tikzpicture}s globally, meaning automatically activated for each \texttt{tikzpicture}. It has to be activated explicitely in the document with \texttt{\textbackslash tikzexternalenable}.

It is possible to control this behavior with 

\begin{verbatim}
\usepackage[globalexternalization=true|false]{stmtikz}
\end{verbatim}

\texttt{globalexternalization=false} is the default. It is used in case \texttt{globalexternalization=true} is not set explicitly.

Global externalization is active until the next \texttt{\textbackslash tikzexternaldisable} in your document.

\subsection{\protect\texttt{stmtikzlibraries}}
\label{sec:usage:preamble:libraries}

% \subsubsection{Description}
% \label{sec:usage:preamble:libraries:description}

This package contains standard libraries commonly required in the creation of \texttt{tikzpicture}s.

% \subsubsection{Options}
% \label{sec:usage:preamble:libraries:options}

\subsection{\protect\texttt{stmtikzstyles}}
\label{sec:usage:preamble:styles}

% \subsubsection{Description}
% \label{sec:usage:preamble:styles:description}

This package contains styles commonly required in the creation of \texttt{tikzpicture}s.

% \subsubsection{Options}
% \label{sec:usage:preamble:styles:options}

\subsection{\protect\texttt{stmtikzexternalization}}
\label{sec:usage:preamble:externalization}

\subsubsection{Description}
\label{sec:usage:preamble:externalization:description}

This package enables the externalization of \texttt{tikzpicture}s.

It requires shell escape. Enabling shell escape is achieved by using the \verb+-shell-escape+ command line option when executing \LaTeX or pdf\LaTeX, e.g.

\begin{verbatim}
pdflatex -shell-escape stmtikz.tex
\end{verbatim}

or with \verb+arara+

\begin{verbatim}
arara: pdflatex: {shell: yes}
\end{verbatim}

\subsubsection{Options}
\label{sec:usage:preamble:externalization:options}

\paragraph{Option \protect\textit{outputfolder}}
\label{sec:usage:preamble:externalization:options:outputfolder}

This option expects a string input. Do not add a slash at the end of the string.

With this option it is possible to define a output folder for all externalized \texttt{tikzpicture}s. The folder location is set relative to the directory of the main \texttt{tex}-file.

\begin{verbatim}
\usepackage[outputfolder=$FOLDERNAME$]{stmtikzexternalization}
\end{verbatim}

The default is \texttt{outputfolder=ZZZ\_TikZ}.

\paragraph{Option \protect\textit{globalexternalization}}
\label{sec:usage:preamble:wholepackage:options:globalexternalization}

This is a boolean option.  Expected values are either \texttt{true} or \texttt{false}. 

By default externalization is not enabled for \texttt{tikzpicture}s globally, meaning automatically activated for each \texttt{tikzpicture}. It has to be activated explicitely in the document with \texttt{\textbackslash tikzexternalenable}.

It is possible to control this behavior with 

\begin{verbatim}
\usepackage[globalexternalization=true|false]{stmtikzexternalization}
\end{verbatim}

\texttt{globalexternalization=false} is the default. It is used in case \texttt{globalexternalization=true} is not set explicitly.

Global externalization is active until the next \texttt{\textbackslash tikzexternaldisable} in your document.

\section{Externalization}

This is a test for the externalization of a \texttt{tikzpicture}.

\begin{figure}[htbp]
\centering
\tikzexternalenable
\tikzsetnextfilename{ExternalizationTest}
\begin{tikzpicture}
\node[inner sep=0pt] (image1) at (0,0){\includegraphics[width=0.5\textwidth,height=0.3\textheight,keepaspectratio]{example-image-a}};
           
\begin{scope}[
  shift={(image1.south west)},
  x={(image1.south east)},
  y={(image1.north west)},
]
  % Help grid and labels
  \pic{stmimagegrid};
\end{scope}
\end{tikzpicture}
\tikzexternaldisable
\end{figure}

\newpage
\appendix

\section{The code}

\subsection{stmtikz.sty}

\lstinputlisting[breaklines,style=texcodestyle]{../../tex/latex/stmtikz/stmtikz.sty}

\newpage
\subsection{stmtikzlibraries.sty}

\lstinputlisting[breaklines,style=texcodestyle]{../../tex/latex/stmtikz/stmtikzlibraries.sty}

\newpage
\subsection{stmtikzstyles.sty}

\lstinputlisting[breaklines,style=texcodestyle]{../../tex/latex/stmtikz/stmtikzstyles.sty}

\newpage
\subsection{stmtikzexternalization.sty}

\lstinputlisting[breaklines,style=texcodestyle]{../../tex/latex/stmtikz/stmtikzexternalization.sty}

\end{document}
