%%%%%%%%%%%%%%%%%%%%%%%%%%%%%%%%%%%%
% Header                           %
%%%%%%%%%%%%%%%%%%%%%%%%%%%%%%%%%%%%
% 
% This is the documentation for the
% stmforest package
%
% Usage
%  - Compile using 'arara -w stmforest.tex'
% 
% Revisions: 2024-11-08 Martin Raedel <martin.raedel@dlr.de>
%                       Initial draft
%               
% Contact:   Martin Raedel,  martin.raedel@dlr.de
%            DLR Lightweight Systems
%          
%                                 __/|__
%                                /_/_/_/  
%            www.dlr.de/sy/en      |/ DLR
%
% Copyright (C) 2024-... DLR Lightweight Systems
% 
%%%%%%%%%%%%%%%%%%%%%%%%%%%%%%%%%%%%
% Content                          %
%%%%%%%%%%%%%%%%%%%%%%%%%%%%%%%%%%%%

% --------------------------- 
% Documentclass
% ---------------------------

\documentclass[%
  type=article,%
  layout=koma,%
  date=true,%
  hyperref=true,%
  listings=true,%
  forest=false,%
]{stmtext}

% ---------------------------
% Build automation
% ---------------------------

% arara: pdflatex: {shell: yes, synctex: yes, interaction: nonstopmode}
% arara: pdflatex: {shell: yes, synctex: yes, interaction: nonstopmode}
% arara: clean: { extensions: [ acn, acr, alg, aux, auxlock, bbl, bcf, blg, dpth, dvi, glg, glo, gls, idx, ilg, ind, ist, kate-swp, lock, lof, log, lol, lot, mw, nlo, out, ps, run.xml, slg, slg*, syg, syg*, syi, syi*, synctex, synctex.gz, tex.backup, tex.kate-swp, toc*, user.adi ] }

% ---------------------------
% Packages
% ---------------------------

\usepackage[T1]{fontenc}
\usepackage[utf8]{inputenc}
\usepackage{enumitem}
\usepackage[
  libraries=true,
  styles=true,
]{stmforest}

% ---------------------------
% Doc info
% ---------------------------

\author{Martin R\"{a}del}
\title{stmforest package description}
\subtitle{Copyright \copyright{} \the\year{} DLR SY STM\\v\formatdate[versiondatestyle]{\DTMToday}}
\date{\today}

%%%%%%%%%%%%%%%%%%%%%%%%%%%%%%%%%%%%
% Document                         %
%%%%%%%%%%%%%%%%%%%%%%%%%%%%%%%%%%%%

\begin{document}

\maketitle

\begin{abstract}
These are the forest definitions for \texttt{stmlatex}. It is build upon the \href{https://ctan.org/pkg/forest}{forest} package.
\end{abstract}

\tableofcontents

\section{Usage - in the preamble}

\subsection{Load the whole \protect\texttt{stmforest} package}

\subsubsection{Description}

This is an interface package which loads \texttt{forest} and definitions commonly required throughout document creation.

By default the package loads

\begin{itemize}[noitemsep]
  \item \verb+stmforestbase.sty+
  \item \verb+stmforestlibraries.sty+
  \item \verb+stmforeststyles.sty+
\end{itemize}

See \autoref{sec:usage:preamble:wholepackage:options} for options to change the default package behavior.

\subsubsection{Options}
\label{sec:usage:preamble:wholepackage:options}

\paragraph{Option \protect\textit{libraries}} 
\label{sec:usage:preamble:wholepackage:options:libraries}

This is a boolean option. Expected values are either \texttt{true} or \texttt{false}. It controls whether to load the standard libraries commonly required.

\begin{verbatim}
\usepackage[libraries=true|false]{stmforest}
\end{verbatim}

\texttt{libraries=true} is the default. It is used in case \texttt{libraries=false} is not set explicitly.

\paragraph{Option \protect\textit{styles}} 
\label{sec:usage:preamble:wholepackage:options:styles}

This is a boolean option. Expected values are either \texttt{true} or \texttt{false}. It controls whether to load the predefined \texttt{forest} styles.

\begin{verbatim}
\usepackage[styles=true|false]{stmforest}
\end{verbatim}

\texttt{styles=true} is the default. It is used in case \texttt{styles=false} is not set explicitly.

\subsection{\protect\texttt{stmforestlibraries}}
\label{sec:usage:preamble:libraries}

% \subsubsection{Description}
% \label{sec:usage:preamble:libraries:description}

This package contains standard libraries commonly required in the creation of \texttt{forest} trees.

% \subsubsection{Options}
% \label{sec:usage:preamble:libraries:options}

\subsection{\protect\texttt{stmforeststyles}}
\label{sec:usage:preamble:styles}

% \subsubsection{Description}
% \label{sec:usage:preamble:styles:description}

This package contains styles commonly required in the creation of \texttt{forest} trees.

\section{Examples}

\subsection{Directory tree}

A simple example of how to create a nice directory tree from the styles defined in \texttt{stmforeststyles}. The result is shown in \autoref{fig:ex:dirtree}.

\begin{figure}[htbp]
\centering
\begin{forest}
pic dir tree,
pic root,
for tree={% folder icons by default; override using file for file icons
    directory,
    fit=band,
    s sep=2.5pt,
},
[folder
    [File1.ext, file]
    [File2.ext, file]
    [...\vphantom{a}, file]
    [subfolder1
        [File3.ext, file]
    ]
    [subfolder2]
    [...\vphantom{a}]
    ]
]
\end{forest}
\caption{Directory tree example}
\label{fig:ex:dirtree}
\end{figure}

\section{Known issues}

\begin{description}[leftmargin=\parindent,labelindent=\parindent,style=nextline]
% 
\item[...]\mbox{}\\[-2.0\baselineskip]
%
\end{description}

\newpage
\appendix

\section{The code}

\subsection{\protect\texttt{stmforest.sty}}

\lstinputlisting[
  style=texpackagedocstyle,%
]{../../tex/latex/stmforest/stmforest.sty}

\subsection{\protect\texttt{stmforestbase.sty}}

\lstinputlisting[
  style=texpackagedocstyle,%
]{../../tex/latex/stmforest/stmforestbase.sty}

\subsection{\protect\texttt{stmforestlibraries.sty}}

\lstinputlisting[
  style=texpackagedocstyle,%
]{../../tex/latex/stmforest/stmforestlibraries.sty}

\subsection{\protect\texttt{stmforeststyles.sty}}

\lstinputlisting[
  style=texpackagedocstyle,%
]{../../tex/latex/stmforest/stmforeststyles.sty}

\end{document}
