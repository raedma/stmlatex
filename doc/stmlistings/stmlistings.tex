%%%%%%%%%%%%%%%%%%%%%%%%%%%%%%%%%%%%
% Header                           %
%%%%%%%%%%%%%%%%%%%%%%%%%%%%%%%%%%%%
% 
% This is the documentation for the
% stmlistings package
%
% Usage
%  - Compile using 'arara -w stmlistings.tex'
% 
% Revisions: 2019-10-27 Martin Raedel <martin.raedel@dlr.de>
%                       Initial draft
%               
% Contact:   Martin Raedel,  martin.raedel@dlr.de
%            DLR Lightweight Systems
%          
%                                 __/|__
%                                /_/_/_/  
%            www.dlr.de/sy/en      |/ DLR
%
% Copyright (C) 2019-... DLR Lightweight Systems
% 
%%%%%%%%%%%%%%%%%%%%%%%%%%%%%%%%%%%%
% Content                          %
%%%%%%%%%%%%%%%%%%%%%%%%%%%%%%%%%%%%

% --------------------------- 
% Documentclass
% ---------------------------

\documentclass[%
  type=article,%
  layout=koma,%
  hyperref=true,%
  date=true,%
  listings=false,%
  tabular=true,%
]{stmtext}

% ---------------------------
% Build automation
% ---------------------------

% arara: pdflatex: {shell: yes, synctex: yes, interaction: nonstopmode}
% arara: pdflatex: {shell: yes, synctex: yes, interaction: nonstopmode}
% arara: clean: { extensions: [ acn, acr, alg, aux, auxlock, bbl, bcf, blg, dpth, dvi, glg, glo, gls, idx, ilg, ind, ist, kate-swp, lock, lof, log, lol, lot, mw, nlo, out, ps, run.xml, slg, slg*, syg, syg*, syi, syi*, synctex, synctex.gz, tex.backup, tex.kate-swp, toc*, user.adi ] }

% ---------------------------
% Packages
% ---------------------------

\usepackage[T1]{fontenc}
\usepackage[utf8]{inputenc}
\usepackage{enumitem}

\usepackage[%
  commands=true,%
  environments=true,%
  styles=true,%
]{stmlistings}

% ---------------------------
% Doc info
% ---------------------------

\author{Martin R\"{a}del}
\title{stmlistings package description}
\subtitle{Copyright \copyright{} \the\year{} DLR SY STM\\v\formatdate[versiondatestyle]{\DTMToday}}
\date{\today}

%%%%%%%%%%%%%%%%%%%%%%%%%%%%%%%%%%%%
% Document                         %
%%%%%%%%%%%%%%%%%%%%%%%%%%%%%%%%%%%%

\begin{document}

\maketitle

\begin{abstract}
These are the listings definitions for \texttt{stmlatex}. It is build upon the \href{https://ctan.org/pkg/listings}{listings} package.
\end{abstract}

\tableofcontents

\section{Usage}% - in the preamble}

\subsection{Load the whole \protect\texttt{stmlistings} package}
\label{sec:usage:preamble:wholepackage}

\subsubsection{Description}
\label{sec:usage:preamble:wholepackage:description}

This is an interface package which loads listings styles and commands commonly required throughout document creation.

By default the package loads

\begin{itemize}[noitemsep]
%   \item \verb+stmlistingsbase.sty+
  \item \verb+stmlistingscommands.sty+
  \item \verb+stmlistingsenvironments.sty+
  \item \verb+stmlistingsstyles.sty+
\end{itemize}

See \autoref{sec:usage:preamble:wholepackage:options} for options to change the default package behavior.

\subsubsection{Options}
\label{sec:usage:preamble:wholepackage:options}

\paragraph{Option \protect\textit{styles}} 
\label{sec:usage:preamble:wholepackage:options:styles}

This is a boolean option. Expected values are either \texttt{true} or \texttt{false}. It controls whether to load the predefined listings styles.

\begin{verbatim}
\usepackage[styles=true|false]{stmlistings}
\end{verbatim}

\texttt{styles=true} is the default. It is used in case \texttt{styles=false} is not set explicitly.

\paragraph{Option \protect\textit{commands}} 
\label{sec:usage:preamble:wholepackage:options:commands}

This is a boolean option. Expected values are either \texttt{true} or \texttt{false}. It enables and disables the loading of specific listings commands.

\begin{verbatim}
\usepackage[commands=true|false]{stmlistings}
\end{verbatim}

\texttt{commands=true} is the default. It is used in case \texttt{commands=false} is not set explicitly.

\paragraph{Option \protect\textit{environments}} 
\label{sec:usage:preamble:wholepackage:options:environments}

This is a boolean option. Expected values are either \texttt{true} or \texttt{false}. It enables and disables the loading of specific listings environments.

\begin{verbatim}
\usepackage[environments=true|false]{stmlistings}
\end{verbatim}

\texttt{commands=true} is the default. It is used in case \texttt{commands=false} is not set explicitly.

\subsection{\protect\texttt{stmlistingsstyles}}
\label{sec:usage:preamble:styles}

\subsubsection{Description}
\label{sec:usage:preamble:styles:description}

\subsubsection{Styles}
\label{sec:usage:preamble:styles:styles}

\subsection{\protect\texttt{stmlistingscommands}}
\label{sec:usage:preamble:commands}

\subsubsection{Description}
\label{sec:usage:preamble:commands:description}

These are commands and shortcuts for listings.

\subsubsection{Command \protect\texttt{\textbackslash inlinecode}}
\label{sec:usage:preamble:commands:inlinecode}

\paragraph{Description}

Create an code snippet in the current line.

\paragraph{Usage} The command arguments are\mbox{}\\

\begin{tabularx}{\linewidth}{ccccX}
\toprule
Nr.  & Entity & Type & Default  & Description\\
\midrule
1 & Delimiter & optional & \verb|+| & Defines beginning and end of the code snippet.\\
2 & Color & optional & \verb|verbgraycolor| & Background color\\
3 & Style & optional & \verb|inlinecodestyle| & Listings style\\
3 & Content & mandatory &  & Code\\
\bottomrule
\end{tabularx}


\paragraph{Examples}

\begin{itemize}
  \item \verb|\inlinecode{Example}|: This is an \inlinecode{Example} for an in line code definition.
  \item \verb|\inlinecode[+][verbgraycolor][inlinecodestyle]{Example}|: This is an \inlinecode[+][verbgraycolor][inlinecodestyle]{Example} for an in line code definition.
  \item \verb|\inlinecode[+][green]{Example}|: This is an \inlinecode[+][green]{Example} for an in line code definition.
  \item \verb#\inlinecode[|]{1+2}#: Example for a different delimiter \inlinecode[|]{1+2} in case the default delimiter is used in the expression.
\end{itemize}

\newpage
\appendix

\section{The code}

\subsection{\protect\texttt{stmlistings.sty}}

\lstinputlisting[
  style=texpackagedocstyle,%
]{../../tex/latex/stmlistings/stmlistings.sty}

% \subsection{\protect\texttt{stmlistingsbase.sty}}
% 
% \lstinputlisting[
%   style=texpackagedocstyle,%
% ]{../../tex/latex/stmlistings/stmlistingsbase.sty}

\newpage
\subsection{\protect\texttt{stmlistingscommands.sty}}

\lstinputlisting[
  style=texpackagedocstyle,%
]{../../tex/latex/stmlistings/stmlistingscommands.sty}

\newpage
\subsection{\protect\texttt{stmlistingsenvironments.sty}}

\lstinputlisting[
  style=texpackagedocstyle,%
]{../../tex/latex/stmlistings/stmlistingsenvironments.sty}

\newpage
\subsection{\protect\texttt{stmlistingsstyles.sty}}

\lstinputlisting[
  style=texpackagedocstyle,%
]{../../tex/latex/stmlistings/stmlistingsstyles.sty}

\end{document}
