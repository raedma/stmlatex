%%%%%%%%%%%%%%%%%%%%%%%%%%%%%%%%%%%%
% Header                           %
%%%%%%%%%%%%%%%%%%%%%%%%%%%%%%%%%%%%
% 
% This is the documentation for the
% stmdate package
%
% Usage
%  - Compile using 'arara -w stmdate.tex'
% 
% Revisions: 2019-10-27 Martin Raedel <martin.raedel@dlr.de>
%                       Initial draft
%               
% Contact:   Martin Raedel,  martin.raedel@dlr.de
%            DLR Lightweight Systems
%          
%                                 __/|__
%                                /_/_/_/  
%            www.dlr.de/sy/en      |/ DLR
%
% Copyright (C) 2019-... DLR Lightweight Systems
% 
%%%%%%%%%%%%%%%%%%%%%%%%%%%%%%%%%%%%
% Content                          %
%%%%%%%%%%%%%%%%%%%%%%%%%%%%%%%%%%%%

% --------------------------- 
% Documentclass
% ---------------------------

\documentclass[%
  type=article,%
  layout=koma,%
  conditionallox=true,%
  conditionalloxnewpage=false,%
  date=true,%
  hyperref=true,%
  index=false,%
  listings=true%
]{stmtext}

% ---------------------------
% Build automation
% ---------------------------

% arara: pdflatex: {shell: yes, synctex: yes, interaction: nonstopmode}
% arara: pdflatex: {shell: yes, synctex: yes, interaction: nonstopmode}
% arara: pdflatex: {shell: yes, synctex: yes, interaction: nonstopmode}
% arara: clean: { extensions: [ acn, acr, alg, aux, auxlock, bbl, bcf, blg, dpth, dvi, glg, glo, gls, idx, ilg, ind, ist, kate-swp, lock, lof, log, lol, lot, mw, nlo, out, ps, run.xml, slg, slg*, syg, syg*, syi, syi*, synctex, synctex.gz, tex.backup, tex.kate-swp, toc*, user.adi ] }

% ---------------------------
% Packages
% ---------------------------

\usepackage[T1]{fontenc}
\usepackage[utf8]{inputenc}
\usepackage{enumitem}

\usepackage[%
  standard=true,% Load a single standard index
  standardname=Testindex,%
  standardtitle={Test\ index},%
  columns=3,
  %nonewpage=true,%
]{stmindex}

\usepackage{hyperref}

% ---------------------------
% Doc info
% ---------------------------

\author{Martin R\"{a}del}
\title{stmindex package description}
\subtitle{Copyright \copyright{} \the\year{} DLR SY STM\\v\formatdate[versiondatestyle]{\DTMToday}}
\date{\today}

%%%%%%%%%%%%%%%%%%%%%%%%%%%%%%%%%%%%
% Document                         %
%%%%%%%%%%%%%%%%%%%%%%%%%%%%%%%%%%%%

\begin{document}

\maketitle

\begin{abstract}
This is an interface to facilitate the creation of a single index for \texttt{stmlatex}-based documents. It is build upon the \href{https://ctan.org/pkg/imakeidx}{imakeidx}\stmindex{imakeidx} package.
\end{abstract}

\tableofcontents

\conditionallistoffigures  % Insert List of Figures if figures are present
\conditionallistoftables   % Insert List of Tables if tables are present
\conditionallistoflistings % Insert List of Listings if listings are present

\section{About}
\stmindex{About}

Using \texttt{stmindex} it is possible to create a single index for a document very conveniently. When you want to add items to the default index, it is not necessary to specify the index name each time. This is also possible with the basic version of \verb+\index+, however, this makes it easier in case you at some point need another index.

It is assumed that for a large part of documents a single index is sufficient. However, \texttt{stmlatex} can also be used as a blueprint to create other indices.

\section{Usage}% - in the preamble}
\stmindex{Usage}

\subsection{Preamble}

\subsubsection{Load the package}

Load the package by adding

\begin{verbatim}
\usepackage[<options>]{stmindex}
\end{verbatim}

to your document preamble. For a list of options, see \autoref{sec:usage:preamble:options}.

\subsubsection{Options}
\label{sec:usage:preamble:options}
\stmindex{Options}

\paragraph{Option \protect\textit{standard}} 
\label{sec:usage:preamble:options:standard}
\stmindex{Options!standard}

This is a boolean option. Expected values are either \texttt{true} or \texttt{false}. It controls whether to load the standard index.

\begin{verbatim}
\usepackage[standard=true|false]{stmindex}
\end{verbatim}

\texttt{standard=true} is the default. It is used in case \texttt{standard=false} is not set explicitly.

\paragraph{Option \protect\textit{name}} 
\label{sec:usage:preamble:options:name}
\stmindex{Options!name}

This is a string option. It controls the name of the standard index files. It only has an affect in case option \texttt{standard=true}.

\begin{verbatim}
\usepackage[standardname=Testindex]{stmindex}
\end{verbatim}

The default is \textit{Index}.

\paragraph{Option \protect\textit{title}} 
\label{sec:usage:preamble:options:title}
\stmindex{Options!title}

This is a string option. It controls the title of the standard index in the document. It only has an affect in case option \texttt{standard=true}.

\begin{verbatim}
\usepackage[standardtitle={Test\ index}]{stmindex}
\end{verbatim}

The default is \textit{Index}. Spaces have to be put by spacing commands explicitly.

\paragraph{Option \protect\textit{intoc}} 
\label{sec:usage:preamble:options:intoc}
\stmindex{Options!intoc}

This is a boolean option. Expected values are either \texttt{true} or \texttt{false}. It controls whether to add the standard index to the table of contents. It only has an affect in case option \texttt{standard=true}.

\begin{verbatim}
\usepackage[intoc=true|false]{stmindex}
\end{verbatim}

\texttt{intoc=true} is the default. It is used in case \texttt{intoc=false} is not set explicitly.

\paragraph{Option \protect\textit{columns}} 
\label{sec:usage:preamble:options:columns}
\stmindex{Options!columns}

This is a integer option. It controls the number of columns used in the index.  It only has an affect in case option \texttt{standard=true}.

\begin{verbatim}
\usepackage[columns=3]{stmindex}
\end{verbatim}

The default is \textit{3}.

\subsubsection{Indexsetup}
\stmindex{indexsetup}

Use the \verb+\indexsetup+ command from \texttt{imakeidx}\stmindex{imakeidx} in the preamble to tune the behavior of your index, e.g. the level, e.g.

\begin{verbatim}
\indexsetup{%
  level=\section*,%
  toclevel=section,%
}
\end{verbatim}

in a class with where \verb+\chapter+ is the highest level, but you want the index as part of an appendix chapter as a section.

\subsection{In the document}

\texttt{stmindex} introduces a convenient shortcut command to print the standard index in case the option \texttt{standard=true}. Use \verb+\printstmindex+ at the location of your choosing inside the document to print the standard index.

In case multiple indices are used, the standard \texttt{imakeidx} commands apply for printing an index.

\section{Create an individual index independent of the standard index}

It is possible to create new indices next to or in replacement of the standard index. It is required to load \texttt{stmtikzexternalization} after the \texttt{\textbackslash makeindex} call.

\begin{lstlisting}[
  breaklines,%
  style=texcodestyle,%
  numbers=left,%
  caption={Setup individual index},
  label=lst:index:individual,
]
% Disable the standard index
\usepackage[
  standard=false
]{stmindex}

% Setup identifiers
\newcommand{\pdKeywordIdxName}{pdkeywordindex}
\newcommand{\pdKeywordIdxTitle}{Peridigm keyword index}

% Macro expansion
\DeclareDocumentCommand{\pdkeywordindex}{%
  O{\pdKeywordIdxName}%style - optional, default in brackets
  m%
}{%
  \index[#1]{#2}%
}

% Make index
\makeindex[%pdkeywords
  name={\pdKeywordIdxName},%
  title={\pdKeywordIdxTitle},%
  intoc,%
  columns=2,%
]

% Print command
\newcommand{\printpdkeywordindex}{\printindex[\pdKeywordIdxName]}

% Must come after \makeindex from imakeidx package, e.g. in stmindex
\usepackage{stmtikzexternalization}
\end{lstlisting}

\section{Known issues}
\stmindex{Issues}

There seems to be a problem with the combination of \texttt{imakeidx} and the externalization of \texttt{tikz} pictures as from \texttt{stmtikz}. Externalization must be activated after the call to \texttt{imakeidx} to make it work. See \url{https://tex.stackexchange.com/q/393697}.

\printstmindex

\newpage
\appendix

\section{The code}

\subsection{\protect\texttt{stmindex.sty}}

\lstinputlisting[
  style=texpackagedocstyle,%
  nolol,%
]{../../tex/latex/stmindex/stmindex.sty}

\subsection{\protect\texttt{stmindexbase.sty}}

\lstinputlisting[
  style=texpackagedocstyle,%
  nolol,%
]{../../tex/latex/stmindex/stmindexbase.sty}

\subsection{\protect\texttt{stmindexstandard.sty}}

\lstinputlisting[
  style=texpackagedocstyle,%
  nolol,%
]{../../tex/latex/stmindex/stmindexstandard.sty}

\subsection{\protect\texttt{stmindexstyles.sty}}

\lstinputlisting[
  style=texpackagedocstyle,%
  nolol,%
]{../../tex/latex/stmindex/stmindexstyles.sty}

\end{document}
