%%%%%%%%%%%%%%%%%%%%%%%%%%%%%%%%%%%%
% Header                           %
%%%%%%%%%%%%%%%%%%%%%%%%%%%%%%%%%%%%
% 
% This is the documentation for the
% stmunits package
%
% Usage
%  - Compile using 'arara stmunits.tex'
% 
% Revisions: 2019-10-27 Martin Raedel <martin.raedel@dlr.de>
%                       Initial draft
%               
% Contact:   Martin Raedel,  martin.raedel@dlr.de
%            DLR Lightweight Systems
%          
%                                 __/|__
%                                /_/_/_/  
%            www.dlr.de/sy/en      |/ DLR
%
% Copyright (C) 2019-... DLR Lightweight Systems
% 
%%%%%%%%%%%%%%%%%%%%%%%%%%%%%%%%%%%%
% Content                          %
%%%%%%%%%%%%%%%%%%%%%%%%%%%%%%%%%%%%

% --------------------------- 
% Documentclass
% ---------------------------

\documentclass[%
  type=article,%
  layout=koma,%
  cleveref=true,%
  conditionallox=true,%
  conditionalloxnewpage=false,%
  date=true,%
  hyperref=true,%
  index=true,%
  listings=true,%
  math=true,%
]{stmtext}

% ---------------------------
% Build automation
% ---------------------------

% arara: pdflatex: {shell: yes, synctex: yes, interaction: nonstopmode}
% arara: pdflatex: {shell: yes, synctex: yes, interaction: nonstopmode}
% arara: pdflatex: {shell: yes, synctex: yes, interaction: nonstopmode}
% arara: clean: { extensions: [ acn, acr, alg, aux, auxlock, bbl, bcf, blg, dpth, dvi, glg, glo, gls, idx, ilg, ind, ist, kate-swp, lock, lof, log, lol, lot, mw, nlo, out, ps, run.xml, slg, slg*, syg, syg*, syi, syi*, synctex, synctex.gz, tex.backup, tex.kate-swp, toc*, user.adi ] }

% ---------------------------
% Packages
% ---------------------------

\usepackage[T1]{fontenc}
\usepackage[utf8]{inputenc}
\usepackage{enumitem}
\usepackage{tabularx}

\usepackage{stmunits}

% ---------------------------
% Doc info
% ---------------------------

\author{Martin R\"{a}del}
\title{stmunits package description}
\subtitle{Copyright \copyright{} \the\year{} DLR SY STM\\v\formatdate[versiondatestyle]{\DTMToday}}
\date{\today}

%%%%%%%%%%%%%%%%%%%%%%%%%%%%%%%%%%%%
% Document                         %
%%%%%%%%%%%%%%%%%%%%%%%%%%%%%%%%%%%%

\begin{document}

\maketitle

\begin{abstract}
These are the unit definitions for \texttt{stmlatex}. It is build upon the \href{https://ctan.org/pkg/siunitx}{siunitx} package.
\end{abstract}

\tableofcontents

\conditionallistoffigures  % Insert List of Figures if figures are present
\conditionallistoftables   % Insert List of Tables if tables are present
\conditionallistoflistings % Insert List of Listings if listings are present

\section{Usage}% - in the preamble}

\subsection{Load the whole \protect\texttt{stmunits} package}
\label{sec:usage:preamble:wholepackage}

\subsubsection{Description}
\label{sec:usage:preamble:wholepackage:description}

This is an interface package which loads the base package and the style choice.

By default the package loads

\begin{itemize}[noitemsep]
  \item \verb+stmunitsbase.sty+
  \item \verb+stmunitsitems.sty+
  \item \verb+stmunitsstyles.sty+
\end{itemize}

See \autoref{sec:usage:preamble:wholepackage:options} for options to change the default package behavior.

\subsubsection{Options}
\label{sec:usage:preamble:wholepackage:options}

\paragraph{Option \protect\textit{styles}} 
\label{sec:usage:preamble:wholepackage:options:styles}

This is a boolean option. Expected values are either \texttt{true} or \texttt{false}. It controls whether to load the predefined styles.

\begin{verbatim}
\usepackage[styles=true|false]{stmunits}
\end{verbatim}

\texttt{styles=true} is the default. It is used in case \texttt{styles=false} is not set explicitly.

\paragraph{Option \protect\textit{style}} 
\label{sec:usage:preamble:wholepackage:options:style}

This option value is the equivalent to \cref{sec:usage:preamble:styles:options:style} and is passed through to the underlying package.

\subsection{\protect\texttt{stmunitsbase}}
\label{sec:usage:preamble:base}

This package loads the underlying base package.% with some default options that can only be set during initialization.

\subsection{\protect\texttt{stmunitsitems}}
\label{sec:usage:preamble:items}

This loads new unit definitions, not initially defined in the base package. See \cref{sec:items} for the definitions.

\subsection{\protect\texttt{stmunitsstyles}}
\label{sec:usage:preamble:styles}

\subsubsection{Description}
\label{sec:usage:preamble:styles:description}

This package contains styles for the representation of units and numbers.

\subsubsection{Options}
\label{sec:usage:preamble:styles:options}

\paragraph{Option \protect\textit{style}} 
\label{sec:usage:preamble:styles:options:style}

This is a string option. A list of possible values is shown in \cref{sec:styles}.

\begin{verbatim}
\usepackage[style=stmdefault]{stmunitsstyles}
\end{verbatim}

\texttt{style=stmdefault} is the default. It is used in case no other style is set explicitly.

\section{Items}
\label{sec:items}

\begin{table}[htbp]
\caption{Additional unit definitions}
\label{tab:items:additionalunits}
\begin{tabularx}{\linewidth}{llXll}
\multicolumn{5}{@{}l@{}}{Length}\\
\verb+\si{\foot}+ & $\si{\foot}$  &&
\verb+\si{\inch}+ & $\si{\inch}$  \\[1.5ex]
\multicolumn{5}{@{}l@{}}{Temperature}\\
\verb+\si{\degreeFahrenheit}+ & $\si{\degreeFahrenheit}$  &&
\verb+\si{\degreeRankine}+ & $\si{\degreeRankine}$  \\[1.5ex]
\multicolumn{5}{@{}l@{}}{Force}\\
\verb+\si{\dyn}+ & $\si{\dyn}$  &&
\verb+\si{\poundforce}+ & $\si{\poundforce}$  \\[1.5ex]
\multicolumn{5}{@{}l@{}}{Pressure}\\
\verb+\si{\barye}+ & $\si{\barye}$  &&
\verb+\si{\psi}+ & $\si{\psi}$  \\[1.5ex]
\multicolumn{5}{@{}l@{}}{Density}\\
\verb+\si{\slug}+ & $\si{\slug}$  \\[1.5ex]
\multicolumn{5}{@{}l@{}}{Energy}\\
\verb+\si{\erg}+ & $\si{\erg}$  \\[1.5ex]
\multicolumn{5}{@{}l@{}}{Misc.}\\
\verb+\si{\promille}+ & $\si{\promille}$  \\
\end{tabularx}
\end{table}

\section{Styles}
\label{sec:styles}

There are different predefined styles available to format your numbers and units.

\subsection{\protect\texttt{default}}
\label{sec:styles:default}

This loads the default style from the underlying base package and does not apply individual styles.

\subsection{\protect\texttt{stmdefault}}
\label{sec:styles:stmdefault}

This loads unit definition style for the stm package. Have a look at the code in \cref{sec:code:styles} for details.

\printstmindex

\newpage
\appendix

\section{The code}
\label{sec:code}

\subsection{\protect\texttt{stmunits.sty}}
\label{sec:code:main}

\lstinputlisting[
  style=texpackagedocstyle,%
]{../../tex/latex/stmunits/stmunits.sty}

\subsection{\protect\texttt{stmunitsbase.sty}}
\label{sec:code:base}

\lstinputlisting[
  style=texpackagedocstyle,%
]{../../tex/latex/stmunits/stmunitsbase.sty}

\subsection{\protect\texttt{stmunitsitems.sty}}
\label{sec:code:items}

\lstinputlisting[
  style=texpackagedocstyle,%
]{../../tex/latex/stmunits/stmunitsitems.sty}

\subsection{\protect\texttt{stmunitsstyles.sty}}
\label{sec:code:styles}

\lstinputlisting[
  style=texpackagedocstyle,%
]{../../tex/latex/stmunits/stmunitsstyles.sty}

\end{document}
