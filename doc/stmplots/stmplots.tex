%%%%%%%%%%%%%%%%%%%%%%%%%%%%%%%%%%%%
% Header                           %
%%%%%%%%%%%%%%%%%%%%%%%%%%%%%%%%%%%%
% 
% This is the documentation for the
% stmtikz package
%
% Usage
%  - Compile using 'arara -w stmtikz.tex'
% 
% Revisions: 2019-10-27 Martin Raedel <martin.raedel@dlr.de>
%                       Initial draft
%               
% Contact:   Martin Raedel,  martin.raedel@dlr.de
%            DLR Lightweight Systems
%          
%                                 __/|__
%                                /_/_/_/  
%            www.dlr.de/sy/en      |/ DLR
%
% Copyright (C) 2019-... DLR Lightweight Systems
% 
%%%%%%%%%%%%%%%%%%%%%%%%%%%%%%%%%%%%
% Content                          %
%%%%%%%%%%%%%%%%%%%%%%%%%%%%%%%%%%%%

% --------------------------- 
% Documentclass
% ---------------------------

\documentclass[%
  type=article,%
  layout=koma,%
  date=true,%
  hyperref=true,%
  listings=true,%
  math=true,%
  plots=false,%
]{stmtext}

% ---------------------------
% Build automation
% ---------------------------

% arara: pdflatex: {shell: yes, synctex: yes, interaction: nonstopmode}
% arara: pdflatex: {shell: yes, synctex: yes, interaction: nonstopmode}
% arara: clean: { extensions: [ acn, acr, alg, aux, auxlock, bbl, bcf, blg, dpth, dvi, glg, glo, gls, idx, ilg, ind, ist, kate-swp, lock, lof, log, lol, lot, mw, nlo, out, ps, run.xml, slg, slg*, syg, syg*, syi, syi*, synctex, synctex.gz, tex.backup, tex.kate-swp, toc*, user.adi ] }

% ---------------------------
% Packages
% ---------------------------

\usepackage[T1]{fontenc}
\usepackage[utf8]{inputenc}
\usepackage{enumitem}

\usepackage[
  libraries=true,
  styles=true,
  externalization=true,
  globalexternalization=false,
  externalizationoutputfolder=ZZY_TikZ,
]{stmplots}

% ---------------------------
% Doc info
% ---------------------------

\author{Martin R\"{a}del}
\title{stmplots package description}
\subtitle{Copyright \copyright{} \the\year{} DLR SY STM\\v\formatdate[versiondatestyle]{\DTMToday}}
\date{\today}

%%%%%%%%%%%%%%%%%%%%%%%%%%%%%%%%%%%%
% Document                         %
%%%%%%%%%%%%%%%%%%%%%%%%%%%%%%%%%%%%

\begin{document}

\maketitle

\begin{abstract}
These are the plots definitions for \texttt{stmlatex}. It is build upon the \href{https://ctan.org/pkg/pgfplots}{pgfplots} package.
\end{abstract}

\tableofcontents

\section{Usage - in the preamble}

\subsection{Load the whole \protect\texttt{stmplots} package}

\subsubsection{Description}

This is an interface package which loads \texttt{pgfplots} and definitions commonly required throughout document creation.

By default the package loads

\begin{itemize}[noitemsep]
  \item \verb+stmplotsbase.sty+
  \item \verb+stmplotslibraries.sty+
  \item \verb+stmplotsstyles.sty+
\end{itemize}

See \autoref{sec:usage:preamble:wholepackage:options} for options to change the default package behavior.

\subsubsection{Options}
\label{sec:usage:preamble:wholepackage:options}

\paragraph{Option \protect\textit{compat}} 
\label{sec:usage:preamble:wholepackage:options:compat}

This option expects a string input. Possible inputs are \texttt{pgfplots} version numbers, e.g.

\begin{verbatim}
\usepackage[compat=1.14]{stmplots}
\end{verbatim}

\texttt{compat=newest} is the default. It is used in case \verb+libraries=$VALUE$+ is not set explicitly.

\paragraph{Option \protect\textit{libraries}} 
\label{sec:usage:preamble:wholepackage:options:libraries}

This is a boolean option. Expected values are either \texttt{true} or \texttt{false}. It controls whether to load the standard libraries commonly required.

\begin{verbatim}
\usepackage[libraries=true|false]{stmplots}
\end{verbatim}

\texttt{libraries=true} is the default. It is used in case \texttt{libraries=false} is not set explicitly.

\paragraph{Option \protect\textit{styles}} 
\label{sec:usage:preamble:wholepackage:options:styles}

This is a boolean option. Expected values are either \texttt{true} or \texttt{false}. It controls whether to load the predefined \texttt{pgfplots} styles.

\begin{verbatim}
\usepackage[styles=true|false]{stmplots}
\end{verbatim}

\texttt{styles=true} is the default. It is used in case \texttt{styles=false} is not set explicitly.

\paragraph{Option \protect\textit{externalization}} 
\label{sec:usage:preamble:wholepackage:options:externalization}

This is a boolean option. Expected values are either \texttt{true} or \texttt{false}. It enables and disables the possibilities for the externalization of \texttt{tikzpicture}s.

\begin{verbatim}
\usepackage[externalization=true|false]{stmplots}
\end{verbatim}

\texttt{externalization=true} is the default. It is used in case \texttt{externalization=false} is not set explicitly.

See the \texttt{stmtikz} package documentation for details.

\paragraph{Option \protect\textit{externalizationoutputfolder}}
\label{sec:usage:preamble:wholepackage:options:outputfolder}

This option expects a string input. Do not add a slash at the end of the string.

With this option it is possible to define a output folder for all externalized \texttt{tikzpicture}s in case \nameref{sec:usage:preamble:wholepackage:options:externalization} has the value \texttt{true}. The folder location is set relative to the directory of the main \texttt{tex}-file.

\begin{verbatim}
\usepackage[externalizationoutputfolder=$FOLDERNAME$]{stmtikz}
\end{verbatim}

The default is \texttt{externalizationoutputfolder=ZZZ\_TikZ}.

\paragraph{Option \protect\textit{globalexternalization}}
\label{sec:usage:preamble:wholepackage:options:globalexternalization}

This is a boolean option.  Expected values are either \texttt{true} or \texttt{false}. 

By default externalization is not enabled for \texttt{tikzpicture}s globally, meaning automatically activated for each \texttt{tikzpicture}. It has to be activated explicitely in the document with \texttt{\textbackslash tikzexternalenable}.

It is possible to control this behavior with 

\begin{verbatim}
\usepackage[globalexternalization=true|false]{stmplots}
\end{verbatim}

\texttt{globalexternalization=false} is the default. It is used in case \texttt{globalexternalization=true} is not set explicitly.

Global externalization is active until the next \texttt{\textbackslash tikzexternaldisable} in your document.

\subsection{\protect\texttt{stmplotslibraries}}
\label{sec:usage:preamble:libraries}

% \subsubsection{Description}
% \label{sec:usage:preamble:libraries:description}

This package contains standard libraries commonly required in the creation of plots.

% \subsubsection{Options}
% \label{sec:usage:preamble:libraries:options}

\subsection{\protect\texttt{stmplotsstyles}}
\label{sec:usage:preamble:styles}

% \subsubsection{Description}
% \label{sec:usage:preamble:styles:description}

This package contains styles commonly required in the creation of plots.

% \subsubsection{Options}
% \label{sec:usage:preamble:styles:options}

\section{Test}

This is a test. Code thankfully taken from

\href{http://pgfplots.net/tikz/examples/fill-between-plots/}{http://pgfplots.net/tikz/examples/fill-between-plots/}

\begin{figure}[htbp]
\centering
\tikzexternalenable
\tikzsetnextfilename{ExternalizationTest}
\begin{tikzpicture}

  \pgfmathdeclarefunction{poly}{0}{\pgfmathparse{-x^3+5*(x^2)-3*x-3}}
  \begin{axis}[
    axis y line = left,
    axis x line = bottom,
    xtick       = {-1.2,2,4.2},
    xticklabels = {$a$,$\zeta$,$b$},
    ytick       = {3},
    yticklabels = {$f(\zeta)$},
    samples     = 160,
    domain      = -1.2:4.2,
    xmin = -2, xmax = 5,
    ymin = -5, ymax = 10,
  ]
    \addplot[name path=poly, black, thick, mark=none, ] {poly};
    \addplot[name path=line, gray, no markers, line width=1pt] {3};
    \addplot fill between[ 
      of = poly and line, 
      split, % calculate segments
      every even segment/.style = {orange!70},
      every odd segment/.style  = {gray!60}
    ];
  \end{axis}
\end{tikzpicture}
\tikzexternaldisable
\end{figure}

\newpage
\appendix

\section{The code}

\subsection{\protect\texttt{stmplots.sty}}

\lstinputlisting[
  style=texpackagedocstyle,%
]{../../tex/latex/stmplots/stmplots.sty}

\subsection{\protect\texttt{stmplotsbase.sty}}

\lstinputlisting[
  style=texpackagedocstyle,%
]{../../tex/latex/stmplots/stmplotsbase.sty}

\subsection{\protect\texttt{stmplotslibraries.sty}}

\lstinputlisting[
  style=texpackagedocstyle,%
]{../../tex/latex/stmplots/stmplotslibraries.sty}

\subsection{\protect\texttt{stmplotsstyles.sty}}

\lstinputlisting[
  style=texpackagedocstyle,%
]{../../tex/latex/stmplots/stmplotsstyles.sty}

\end{document}
