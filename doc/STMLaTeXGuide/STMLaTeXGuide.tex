%%%%%%%%%%%%%%%%%%%%%%%%%%%%%%%%%%%%
% Header                           %
%%%%%%%%%%%%%%%%%%%%%%%%%%%%%%%%%%%%
% 
% This is the documentation for the
% stmmath package
%
% Usage
%  - Compile using 'arara stmmath.tex'
% 
% Revisions: 2019-10-27 Martin Raedel <martin.raedel@dlr.de>
%                       Initial draft
%               
% Contact:   Martin Raedel,  martin.raedel@dlr.de
%            DLR Lightweight Systems
%          
%                                 __/|__
%                                /_/_/_/  
%            www.dlr.de/fa/en      |/ DLR
%
% Copyright (C) 2019-... DLR Lightweight Systems
% 
%%%%%%%%%%%%%%%%%%%%%%%%%%%%%%%%%%%%
% Content                          %
%%%%%%%%%%%%%%%%%%%%%%%%%%%%%%%%%%%%

% --------------------------- 
% Documentclass
% ---------------------------



\documentclass[%
  type=article,%
  layout=koma,%
  page=false,%
  hyperref=true,%
  cleveref=true,%
  conditionallox=true,%
  conditionalloxnewpage=true,%
  date=true,%
  glossaries=true,%
  index=true,%
  math=true,%
  listings=true%
]{stmtext}

% ---------------------------
% Build automation
% ---------------------------

% arara: pdflatex: {shell: yes, synctex: yes, interaction: nonstopmode}
% arara: pdflatex: {shell: yes, synctex: yes, interaction: nonstopmode}
% arara: clean: { extensions: [ acn, acr, alg, aux, auxlock, bbl, bcf, blg, dpth, dvi, glg, glo, gls, idx, ilg, ind, ist, kate-swp, lock, lof, log, lol, lot, mw, nlo, out, ps, run.xml, slg, slg*, syg, syg*, syi, syi*, synctex, synctex.gz, tex.backup, tex.kate-swp, toc*, user.adi ] }

% ---------------------------
% Packages
% ---------------------------

\usepackage[T1]{fontenc}
\usepackage[utf8]{inputenc}
\usepackage{stmcolors}

\usepackage[autostyle=true,german=quotes]{csquotes}
\usepackage{tabularx}
\usepackage{xspace}

%\newcommand{\package}[2]{\subsection*{#1}\unskip(by #2)\par\medskip\noindent\ignorespaces}
\newcommand{\package}[1]{\subsection*{#1}\par\medskip\noindent\ignorespaces}
\newcommand{\stmlatex}{STM\LaTeX\xspace}

% ---------------------------
% Doc info
% ---------------------------

\author{Martin R\"{a}del}
\title{\stmlatex}
\subtitle{%
Overview and Installation Guide\\
Copyright \copyright{} \the\year{} DLR SY STM\\v\formatdate[versiondatestyle]{\DTMToday}\\
{\bigskip\small\normalfont Thankfully inspired by the RM-\LaTeX package guide.}
}
\date{\today}

%%%%%%%%%%%%%%%%%%%%%%%%%%%%%%%%%%%%
% Document                         %
%%%%%%%%%%%%%%%%%%%%%%%%%%%%%%%%%%%%

\begin{document}

% Titlepage
% ---------------------------
\maketitle

% Abstract
% ---------------------------

\begin{abstract}
\stmlatex is a template collection for \LaTeX{} documents. All features and templates are created due to personal interest and are subject to change. All packages are tested using MiKTeX 2.9. However, they are considered as \enquote{as-is} and without any warranty. For questions, suggestion and feedback, please contact the author of the particular package.

\begin{center}
\color{red}
\bfseries
\stmlatex requires an up to date \LaTeX{} installation.
\end{center}
\end{abstract}

% TOC
% ---------------------------

\tableofcontents

% Content
% ---------------------------

\section{An Overview}

The following templates and tools are available in \stmlatex. Further information on each package can be found in the package documentation.

\package{stmcolors}
Color definitions.

\package{stmconditionallox}
Conditional lists of float environments.

\package{stmdate}
Date definitions.

\package{stmglossaries}
Acronyms and symbol definitions.

\package{stmhyperref}
Definitions for hypertext.

\package{stmindex}
Index definitions.

\package{stmlanguage}
Language definitions.

\package{stmlistings}
Listings definitions.

\package{stmmath}
Math definitions.

\package{stmplots}
Plot definitions.

\package{stmtext}
Documentclass for structural mechanics document preparation.

\package{stmtikz}
Drawing definitions.

\package{stmunits}
Units definitions.

\section{Installation}

The \stmlatex package can be downloaded or cloned from \url{https://github.com/raedma/stmlatex}. The installation procedure for the \stmlatex package on different operating systems is explained in the following.

% Windows
\subsection{Windows: MiKTeX}

In order to correctly install \stmlatex on a Windows system with MikTeX the following steps have to be performed. 

\begin{enumerate}
  \item Copy \stmlatex to any desired path, e.\,g. to \texttt{D:\textbackslash My-LaTeX-Templates\textbackslash STMLaTeX\textbackslash}. It is important that you do not change the directory structure inside the package.
  \item Open the MikTeX Console GUI and add the \stmlatex package root path to the MikTeX root directory list. Currently, \texttt{Settings} $\rightarrow$ \texttt{Directories} $\rightarrow$ \texttt{Add} (Plus symbol). Click OK.
\end{enumerate}

% Linux
\subsection{Linux: TeXLive}

For an installation with TeXLive a user specific installation and an installation for all users is possible. Both possibilities are explained in the following subsections.

% Installation by an User
\subsubsection{Installation by an User}
\begin{enumerate}		
  \item Unpack the files into the user's \texttt{texmf/} directory. It is important that you do not change the directory structure.
  \item Update the filename database by executing \texttt{mktexlsr}.
  \item If you use Debian, you have to update the config files by running \texttt{update-updmap}.
  \item Final you have to update the font maps. This is be done by calling \texttt{updmap}.
\end{enumerate}

% Installation for all Users
\subsubsection{Installation for all Users}
Please note: if yout want to install the \stmlatex package for all users, administrator rights are necessary.
\begin{enumerate}
  \item Unpack the files to \texttt{/usr/local/share/texmf/}.
  \item Update the filename database by executing \texttt{sudo mktexlsr}.
  \item If you use Debian, you have to update the config files by \texttt{sudo update-updmap}.
  \item Final you have to update the font maps by executing \texttt{sudo updmap-sys}.
\end{enumerate}

% Getting started
\section{Getting started}

All files below \texttt{doc/} are self-running examples.

\end{document}
